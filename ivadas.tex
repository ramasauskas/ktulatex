\section*{Įvadas}
Įvado pradžioje nurodoma, kokiai studijų programai ir specializacijai priklauso
darbas. Dirbtinio intelekto studijų programos studentai specializacijų neturi
(todėl nurodo tik studijų programą), Informatikos studijų programos studentai
nurodo vieną iš trijų variantų: Interneto informatika (specializacija),
Multimedijų sistemos (specializacija), Asmeninis modulių rinkinys. 

Toliau supažindinama su darbo specifika, aktualumu, išdėstomi tikslai bei uždaviniai,
praktinė darbo reikšmė, aptariama dokumento struktūra. Šiame skyriuje apie
darbą kalbama abstrakčiai, nederėtų pateikti nuorodų į kitus šaltinius (1 – 2
lapai). 

\subsection*{Darbo problematika ir aktualumas}
Apibrėžiama darbo problematika ir aptariamas aktualumas. Šiame poskyryje taip pat nurodoma su darbu susijusi sritis, praktinė darbo reikšmė.
\subsection*{Darbo tikslas ir uždaviniai}
Suformuluojamas pagrindinis darbo tikslas, kuris išskaidomas į kelis uždavinius (3 – 6 uždaviniai). Išvados dokumento pabaigoje formuluojamos uždavinių pagrindu.
\subsection*{Darbo struktūra}
Aptariama dokumento struktūra. Nurodoma kiek ir kokių skyrių dokumente yra ir kokia informacija juose pateikiama.
\addcontentsline{toc}{section}{Įvadas}
