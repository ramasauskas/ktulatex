\section{Projektas}
Aprašoma sistemos/įrankio/paslaugos projektavimo stadija, pateikiama detali
specifikacija (8 – 12 lapai). Apibrėžiama kuriamo produkto vizija (koncepcija).
Pagrindiniai projektiniai sprendimai turėtų būti pateikti grafiškai,
prisilaikant notacijų ir pan. \subsection{Reikalavimų specifikacija} Pateikiami
aiškiai suformuluoti
funkciniai reikalavimai kuriamai sistemai/įrankiui/paslaugai. Reikalavimai
formuluojami užsakovo pateiktos techninės užduoties ir atliktos analizės
pagrindu.
\subsubsection{Funkciniai reikalavimai}
Funkciniai reikalavimai aprašomi panaudojimo atvejų diagramomis. Kiekvieno
panaudojimo atvejo specifikacijoje turi būti nurodyta: vartotojas/aktorius,
aprašas, prieš sąlyga, sužadinimo sąlyga, po-sąlyga (žr. Volere šabloną ar kt.
notaciją). \subsubsection{Nefunkciniai reikalavimai} Pateikiami aiškiai
suformuluoti
nefunkciniai reikalavimai kuriamai sistemai/įrankiui/paslaugai. Reikalavimai
formuluojami užsakovo pateiktos techninės užduoties ir atliktos analizės
pagrindu. Paprastai aptariami reikalavimai: sistemos/įrankio/paslaugos
išvaizdai (angl. look and feel), panaudojamumui (angl. usability), vykdymo
charakteristikoms (angl. performance), veikimo sąlygoms (angl. operational),
sistemos/paslaugos priežiūrai (angl. maintainability and portability), saugumui
(angl. security) (žr. Volere šabloną ar kitą notaciją).
\subsubsection{Sistemos veiklos logika}
\subsubsection{Taikomo dirbtinio intelekto sprendimo aprašymas}
\subsubsection{Teorinis modelio aprašas}
\subsubsection{Prognozavimas realiu laiku}
\subsubsection{Duomenų modelio specifikacija}
