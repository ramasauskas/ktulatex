\section{Realizacija ir testavimas}
\subsection{Sistemos realizacijos modelis}
\subsection{Taikomo dirbtinio intelekto sprendimo analizė}
\subsubsection{Duomenų rinkinys}
\subsubsection{Modelio tikslumo įvertinimas}
\subsubsection{Modelio tikslumo rezultatai}
Buvo realizuotas modelis, paremtas \cite{attpaper} straipsniu. Rezultatai pateikiami \ref{tab:findparam} lentelėje.
Paveiklas pateikiamas \ref{fig:ktu_logo} pav.
\begin{table}[H]
\caption{Modelio rezultatai su skirtingais hiperparametrais}
\centering
\begin{tabular}{ccccc}
\toprule
\multicolumn{2}{c}{\textbf{Prognozavimo langas $W'$}} & \multicolumn{3}{c}{\textbf{Enkoderių blokai $U$}} \\
\cmidrule(lr){1-2} \cmidrule(lr){3-5}
\textbf{Ilgis} & \textbf{F1} & \textbf{Blokų kiekis} & \textbf{F1} & \textbf{Vid. apmokymo sparta (it./s)} \\
\midrule
1 & 0,9542 & 1 & 0,9628 & 110,21\\
2 & \textbf{0,9628} & 2  & 0,9684 & 63,43 \\
3 & 0,9620 & 3  & 0,9711 & 43,65\\
4 & 0,9623 & 4  &\textbf{0,9782} & 31,65 \\
\bottomrule
\end{tabular}
\label{tab:findparam}
\end{table}

\begin{figure}[H]
\centering
\includegraphics{ktu.pdf}
\caption{KTU logotipas}
\label{fig:ktu_logo}
\end{figure}

\begin{equation}
\text{Attention}(Q, K, V) = \text{softmax}\left(\frac{QK^T}{\sqrt{d_k}}\right) V
\end{equation}
\subsection{Sistemos realizacijos testavimas}
\subsection{Sukurtos sistemos trūkumai, apribojimai bei tolimesnio plėtojimo galimybės}
